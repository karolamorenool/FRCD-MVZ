\documentclass[12pt]{article}
\usepackage{graphicx} 
\usepackage[spanish]{babel}	
\usepackage[utf8]{inputenc}
\usepackage[colorlinks=true, citecolor=blue]{hyperref}
\usepackage[
backend=biber,	% Backend para las referencias (no modificar)
style=apa,  % Estilo alfabético 
sortcites,		% Para tener ordenadas las citas
url=true, 		% Para que aparezca o no la url
]{biblatex}

\addbibresource{biblio.bib}
%opening

\title{Diseño  de jardines comunitarios en arquitectura}
\author{Karola Moreno, Arelis Verduga,  Irene Zambrano}
\date{\today}

%encabezado
\usepackage{fancyhdr}
\pagestyle{fancy}
\cfoot{}
\rhead{\thepage}
\chead{}
\lhead{}
\begin{document}
%\maketitle
\begin{titlepage}
	\centering
	{\includegraphics[scale=0.5]{logouteq.png}\par}
	\vspace{1cm}
	{\large \textsc{Facultad de Ciencia de la Ingeniería, Universidad Técnica Estatal de Quevedo}\par}
	\vfill
	{\large \textsc{Diseño  de jardines comunitarios en arquitectura}\par}
	\vfill
	{\large Karola Moreno, Arelis Verduga, Irene Zambrano \par}
	\vspace{1cm}
	{\large Fundamentos De Redacción Científica \par}
	\vspace{1cm}
	{\large Ing. Gleiston Guerrero \par}
    \vfill
    {\large 28 de enero 2024}
\end{titlepage}
\begin{abstract}
texto de relleno

\end{abstract}
\newpage

\section{Instroducción}
Hoy en día, más de la mitad de la población mundial reside en entornos urbanos de concreto y asfalto, aunque estas ciudades parecen fuertes y modernas, tienen vulnerabilidades, siendo la alimentación su punto crítico. \\
Esta fragilidad es menos sostenible que su crecimiento demográfico; se espera que para 2050, dos de cada tres personas vivan en áreas urbanas \textcite{martinezvargas0}. Este cambio demográfico plantea desafíos como la desigualdad social, la violencia, la inequidad alimentaria, la sobreexplotación agrícola y la reducción de recursos no renovables. \\

En Ecuador, el paisajismo no ha sido debidamente explorado ni abordado abiertamente hasta el momento. Esto se debe a la escasa difusión y conocimiento acerca de la relevancia de este tema en la sociedad. Aunque hay una preocupación existente por la preservación del medio ambiente, lo cual ha llevado a que la gente cuide de ciertos espacios, aún falta conciencia sobre la importancia de las contribuciones individuales para lograr un impacto significativo. \\

En el ámbito de la puesta en marcha, estas consideraciones convencionales acerca del diseño de jardines pueden tener un impacto en la planificación y ejecución de proyectos paisajísticos en Ecuador. Dada la limitada exploración del paisajismo en el país, resulta crucial incorporar estos principios con el fin de estimular la integración armoniosa de plantas y elementos en el diseño de jardines comunitarios. Asimismo, se requiere crear conciencia entre la población acerca de la relevancia de su contribución individual para preservar y fomentar la biodiversidad en estos espacios, lo cual contribuirá al equilibrio y estabilidad del sistema paisajístico. \\
\newpage
\section{Sistemas propuesto}
\subsection{Desarrollo sostenible}
Para alcanzar este desarrollo, se requiere la movilización y participación de fuerzas considerables. Por esta razón, se promueve la implementación de diversos sistemas Gomez I. en 2020, como a necesidad de involucrar a actores clave y activar recursos significativos se convierte en un punto esencial para impulsar el progreso sostenible. Este enfoque estratégico busca no solo fomentar la colaboración entre diferentes sectores, sino también optimizar la eficiencia de los sistemas propuestos, propiciando así un impacto positivo y sostenido en el desarrollo, como se observa en la siguiente tabla: 
\begin{table}[h]
	\centering
	\caption{Diferentes sectores con su propósito.}
\begin{tabular}{|c|p{10cm}|}
	\hline
	&  \\
	\hline
	Político & Asegure la participación efectiva en los procesos de resolución, promoviendo la inclusión y la representatividad en la toma de decisiones.  \\
	\hline
	Económico & Capaz de generar excedentes y conocimientos técnicos, proporcionando así continuidad y confianza en el desarrollo sostenible. \\
	\hline
	Social & Que pueda abordar y ofrecer soluciones a los problemas derivados de tensiones causadas por un desarrollo desequilibrado, fomentando la equidad y la cohesión social.  \\
	\hline
	Productivo & Que al mismo tiempo que impulsa el desarrollo, respeta la obligación fundamental de preservar la base ecológica, priorizando la sostenibilidad ambiental. \\
	\hline
	Tecnológico & Siempre en búsqueda de nuevas soluciones, fomentando la innovación y la adopción de tecnologías que contribuyan al progreso sostenible. \\
	\hline
	Internacional & Que fomente modelos sostenibles de comercio y financiación, promoviendo la cooperación global y la solidaridad en la construcción de un futuro sostenible.  \\
	\hline
	Administrativo & Flexible y capaz de autocorregirse, buscando la eficiencia y adaptabilidad para enfrentar los desafíos y cambios que puedan surgir en el camino hacia el desarrollo sostenible.  \\
	\hline
\end{tabular}	
\end{table} 
\newpage
\subsection{Economía circular}
La economía circular, en primera instancia, representa un paradigma cuyo propósito central es impulsar la prosperidad económica, salvaguardar el medio ambiente y prevenir la contaminación, con el objetivo fundamental de facilitar el desarrollo sostenible. En las últimas décadas, ha ocurrido un aumento significativo en el consumo de materias primas, comprometiendo la capacidad de producción del planeta. Si este patrón no se modifica, la disponibilidad de recursos y la calidad ambiental se verán afectadas, intensificando el actual desequilibrio presente en la naturaleza (Morato, 2017). \\
La Economía Circular (EC) posibilita abordar los desafíos asociados al crecimiento económico y productivo contemporáneo, al fomentar un ciclo continuo que involucra la extracción, transformación, distribución, uso y recuperación de materiales y energía en productos y servicios, tal como se ilustra en la siguiente imagen. \\
\begin{figure}[h]
	\centering
	\includegraphics[width=0.4\linewidth]{figuras/economiacircular}
	\caption{Economía circular}
	\label{fig:economiacircular} 
\end{figure} 
\newpage
Simplificando, la economía circular puede facilitar la construcción de la Agricultura Urbana (AU). Esto se logra mediante la creación de un sistema con una entrada inicial que incorpora procesos de reutilización de la materia prima. Un ejemplo sencillo se observa en la siembra, donde se obtiene un producto generando al mismo tiempo un residuo. Este residuo se aprovecha al convertirlo en compost, el cual se utiliza como suplemento nutritivo en el suelo para futuras siembras. Además, en la agricultura urbana, se emplean insumos de segunda mano, conocidos coloquialmente como materiales reciclados. La economía circular se implementa intrínsecamente en todo el proceso, desde el diseño y la construcción hasta el funcionamiento de la AU. \\
\subsection{Agricultura urbana}
En un horizonte caracterizado por la crisis energética y los límites de capacidad del planeta, resulta imperativo reconsiderar el modelo urbano existente. Los jardines comunitarios o huertos urbanos, en el contexto de la Agricultura Urbana (AU), se erigen como una estrategia para integrar la naturaleza en el tejido urbano. Esta integración no solo promueve la eficiencia del metabolismo urbano, sino que también fomenta el aumento de la diversidad biológica en las áreas urbanas. En un contexto donde la presión sobre los recursos y la necesidad de sostenibilidad son apremiantes, la adopción de estos espacios colaborativos representa un enfoque esencial para modelar ciudades más resilientes y equitativas. La conexión entre la Agricultura Urbana y los jardines comunitarios no solo responde a la crisis actual, sino que también sienta las bases para entornos urbanos más sostenibles y armoniosos en el futuro. \\
La Agricultura Urbana (AU) está experimentando un crecimiento significativo en diversas ciudades industriales como Londres, Ámsterdam, Malé, Berlín, Hong-Kong, Buenos Aires, Calgary, entre otras. Este fenómeno se manifiesta a través de la adopción de prácticas de microagricultura o agricultura urbana, que se extienden de manera notoria y positiva en espacios públicos como parques, jardines residenciales, escuelas, techos y aceras, tal como se aprecia en las siguientes imágenes. \\
\newpage
\begin{figure}[h]
	\centering
	\includegraphics[width=0.4\linewidth]{figuras/agriculutarurbana}
	\caption{Agricultura Urbana, en Asia y Europa}
	\label{fig:agriculutarurbana}
\end{figure}
La agricultura urbana, con una larga tradición en Asia y Europa, se ha practicado durante muchos años, inicialmente motivada por la baja calidad del transporte y la cercanía de los productos al consumidor. Esta técnica, que solía ser una forma de subsistencia, aún persiste como tal para muchas personas de bajos recursos. Además de contribuir a la supervivencia, la agricultura urbana añade frescura, variedad y un disfrute estético mayor a la oferta alimentaria en entornos urbanos. \\
\begin{figure}[h]
	\centering
	\includegraphics[width=0.4\linewidth]{figuras/aurbana}
	\caption{Agricultura Urbana, durante la pandemia Covid 19}
	\label{fig:aurbana}
\end{figure}
La pandemia de la Covid-19 ha resaltado la limitación del dinero para satisfacer todas las necesidades, especialmente en el acceso a alimentos en áreas urbanas. Las restricciones y el aislamiento han evidenciado la fragilidad de la rutina diaria, generando un debate sobre lo esencial en la vida urbana. La respuesta ha sido la aparición de la agricultura urbana, que cultiva alimentos saludables, fortalece la comunidad y revive prácticas como el "trueque" entre vecinos. Este cambio refleja la conciencia de "blindarnos" ante la pandemia mediante el consumo de alimentos orgánicos, aprovechando espacios en hogares y proporcionando beneficios que van desde la seguridad alimentaria hasta la resiliencia climática, fomentando una convivencia armoniosa y la reconexión con la naturaleza en momentos de incertidumbre. \\
\newpage
\section{Materiales y métodos}
\subsection{Metodología para la integración de la economía circular en un jardín comunitar}
Esta metodología implica realizar una revisión documental fundamentada en la investigación "¿Qué es la Economía Circular?" de Fernández (2010). Se abordan los seis principios fundamentales de la economía circular, de los cuales se seleccionarán tres para su interpretación en este estudio.
\begin{table}[h]
	\centering
	\caption{Diferentes sectores con su propósito.}
	\begin{tabular}{|c|p{10cm}|}
		\hline
		&  \\
		\hline
		Zero waste o desperdicio cero & La perspectiva se enfoca en el uso y reutilización de artículos desechables al reintroducirlos al inicio de la cadena de valor para su reproducción en la misma línea de producción o en una nueva, renovando el ciclo. El propósito es prolongar la vida útil de los productos, aprovechando sus propiedades para insumos similares o transformándolos en otros bienes útiles. \\
		\hline
		Diseño ecológico & Para preservar los bienes en el sistema el mayor tiempo posible, las empresas deben promover el diseño ecológico desde la concepción del producto. Este enfoque implica considerar los posibles impactos ambientales desde la fase inicial de creación, especialmente en productos que contengan baterías, piezas metálicas o componentes electrónicos, para garantizar un diseño que facilite la reutilización o el reciclaje responsable. \\
		\hline
		Reparación de insumos & Es esencial cambiar la mentalidad de desechar en lugar de reparar, superando la obsolescencia programada que beneficia a grandes compañías. La propuesta consiste en dar una segunda vida a productos dañados, desplazando la preferencia por lo nuevo en favor de productos de segunda mano o reciclados. \\
		\hline
	\end{tabular}
\end{table} 
\newpage
\printbibliography[heading=bibintoc]

\end{document}

